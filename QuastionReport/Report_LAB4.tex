\documentclass[12pt, a4paper]{article}

\usepackage[utf8]{inputenc}
\usepackage[T2A]{fontenc}
\usepackage[russian]{babel}
\usepackage[dvips]{graphicx}

\usepackage[oglav,spisok,boldsect,eqwhole,figwhole,hyperref,hyperprint,remarks,greekit]{./style/fn2kursstyle}
\graphicspath{{./style/}{./figures/}}
\usepackage{float}
\usepackage{multirow}
\usepackage{supertabular}
\usepackage{multicol}
\usepackage{hhline}
\usepackage{listings}
\usepackage{color}
\usepackage{adjustbox}
\usepackage{amsmath}
\usepackage{verbatim}
\usepackage{amsfonts}
%\usepackage{mathabx}
\usepackage{graphicx}
\usepackage{subcaption}

\definecolor{dkgreen}{rgb}{0,0.6,0}
\definecolor{gray}{rgb}{0.5,0.5,0.5}
\definecolor{mauve}{rgb}{0.58,0,0.82}

\lstset{frame=tb,
	language=C++,
	aboveskip=1mm,
	belowskip=1mm,
	showstringspaces=false,
	columns=flexible,
	basicstyle={\small},
	numbers=left,
	numberstyle=\tiny\color{gray},
	keywordstyle=\color{red},
	commentstyle=\color{dkgreen},
	stringstyle=\color{mauve},
	breaklines=true,
	breakatwhitespace=true,
	tabsize=2
}
\title{Численное решение краевых задач
	для двумерного уравнения Пуассона \\ Варианты 5, 16}


%\authorfirst{О.\,Д.~Климов}
%\authorsecond{О.\,Д.~Климов} TODO: прописать команды в style.sty

%\supervisor{С.\,А.~Конев}
\supervisor{ }
\group{ФН2-61Б}
\date{2024}

%\renewcommand{\vec}[1]{\text{\mathversion{bold}${#1}$}}%{\bi{#1}}
\newcommand\thh[1]{\text{\mathversion{bold}${#1}$}}
\renewcommand{\labelenumi}{\theenumi)}
\renewcommand{\labelenumi}{\theenumi)}

\newcommand{\opr}{\textbf{\underline{{Опр.}}}\quad}
\newcommand{\theorem}{\textbf{\underline{{Теор.}}}\quad}
\renewcommand{\phi}{\varphi}
\renewcommand{\k}[1]{\textbf{\textit{#1}}}
\newcommand{\widecheck}[1]{\check{#1}}

\newcounter{mycounter}
\newcommand{\quastion}[1]{%
	\stepcounter{mycounter}%
	\textbf{\themycounter}.  %
	\textbf{\textit{#1}}
	
}
\newcommand{\down}[1]{\widecheck{#1}}
\newcommand{\pon}[1]{\mathop {#1}\limits^ \circ}
\newcommand{\rusg}{\text{Г}}

\begin{document}
	\maketitle
	\section{Ответы на контрольные вопросы}
	
	\quastion{Оцените число действий, необходимое для перехода на
	следующий слой по времени методом переменных направлений.}
	
	% ### Первый вариант ответа на вопрос
	
	
	%Имеем, что при фиксированном $j$ прогонка по направлению $x_1$ выполняется за
	%\begin{equation} \label{eq1}
%		\frac{2}{\tau} y^{k + 1 / 2} -  \Lambda_1 y^{k + 1 / 2} = \frac{2}{\tau} y^k +  \Lambda_2 y^k + \phi,
	%\end{equation}
	
	%\begin{equation} \label{eq2}
	%	\frac{2}{\tau} y^{k + 1} -  \Lambda_2 y^{k + 1} = \frac{2}{\tau} y^{k + 1 / 2} +  \Lambda_1 y^{k + 1 /2} + \phi.
	%\end{equation}
	
	%Обозначим
	%\begin{equation*} 
	%	F(y) = \frac{2}{\tau} y +  \Lambda_2 y + \phi, \quad F^{k}_{ij} = F(y^k_{ij}),
	%\end{equation*}
	%\begin{equation*} 
	%	\hat{F}(y) = \frac{2}{\tau} y +  \Lambda_1 y + \phi, \quad \hat{F}^{k}_{ij} = \hat{F}\bigl(y^{k + 1/2}_{ij}\bigr).
	%\end{equation*}
	
	%	Заметим, что эти величины вычисляются явным образом. Обозначая через $\Omega_{i,j} = \xi(x_{1,i},\,x_{2,j})$ значения искомой функции в граничных узлах области, преобразуем уравнения~\eqref{eq1} и~\eqref{eq2}:
	%\begin{equation*} 
	%	\frac{1}{h_1^2} y^{k + 1 / 2}_{i - 1,j} - 2\biggl(\frac{1}{h^2_1} + \frac{1}{\tau}\biggr) y_{ij}^{k + 1 / 2}  + \dfrac{1}{h_1^2} y_{i + 1,j}^{k + 1 / 2} = -F_{ij}^k
	%\end{equation*}
	%\begin{equation} \label{eq3}
	%	u_{0,j} = \Omega_{0,j}, \quad u_{N_1,j} = \Omega_{N_1,j}, \quad j = 1,\,2,\,\dots,\,N_2 - 1.
	%\end{equation}
	
	%\begin{equation*} 
%		\frac{1}{h_2^2} y^{k + 1}_{i,j - 1} - 2\biggl(\frac{1}{h^2_2} + \frac{1}{\tau}\biggr) y_{ij}^{k + 1}  + \dfrac{1}{h_2^2} y_{i,j + 1}^{k + 1} = -\hat{F}_{ij}^{k + 1 / 2}
	%\end{equation*}
%	\begin{equation} \label{eq4}
%		u_{i,0} = \Omega_{i,0}, \quad u_{i,N_2} = \Omega_{i,N_2}, \quad i = 1,\,2,\,\dots,\,N_1 - 1.
	%\end{equation}
	
%	Пусть значение $y^k$ известно из начальных данных либо вычислено на предыдущем шаге по времени. Сначала вычисляем значения $F^k$ в каждой внутренней точке, затем поочередно для каждого индекса $j = 1,\,2,\,\dots,\,N_2 - 1$ вдоль направления $x_1$ решаем методом прогонки систему~\eqref{eq3} и находим значения $y^{k + 1 / 2}$. После этого вычисляем значения $\hat{F}^{k + 1 / 2}$, решаем систему~\eqref{eq4} для каждого индекса $i = 1,\,2,\,\dots,\,N_1 - 1$ вдоль направления $x_2$, находим $y^{k + 1}$. При переходе на каждый новый временной слой процедура повторяется, то есть направление счета все время чередуется.
	
%	Получаем $O(N_1)$ операций, следовательно нахождение всех $y_{i,j}^{k + 1 / 2}$ будет выполнено за $O(N_1N_2)$ операций. Нахождение $y_{i,j}^{k + 1}$ производится также за $O(N_1N_2)$ операций. Таким образом количество операций будет равным $O(2N_1N_2)$.



	% ### Второй вариант ответа на вопрос
	
	Имеем схему 
	\begin{equation} \label{eq1}
		2 \frac{y^{k + 1 / 2} - y^{k}}{\tau} = \Lambda_1 y^{k}  +  \Lambda_2 y^{k+1/2} + \phi,
	\end{equation}
	которая является неявной по направлению $x1$ и явной по направлению $x2$. Тогда подряд для каждого индекса $j = 1, 2, ... N_2 - 1$ вдоль направления $x_1$ решаем систему методом прогонки. 
	Известно, что число операций деления и умножения в методе прогонки равняется $5n$. Получаем, что в нашем случае число действий будет $5N_1(N_2 - 2)$. 
	
	На втором этапе данная схема:
	\[
	2 \frac{y^{k+1} - y^{k+ \frac{1}{2}} }{ \tau} = \Lambda_1  y^{k+\frac{1}{2}} + \Lambda _2 y^{k+1} + \varphi
	\]
	является уже неявной по направлению $x_2$ и явной по направлению $x_1$. Тогда соответственно для каждого индекса $i = 1, 2, ..., N_1 -1$ вдоль направления $x_2$ также решаем систему методом прогонки. Получаем, что число арифметических операций будет $5N_2(N_1 - 2)$.
	
	
	Как итог, для перехода на следующий временной слой требуется:
	\[
	5N_1(N_2 - 2) + 5N_2(N_1 - 2) = 5 N_1 N_2 - 10 N_1 + 5 N_2 N_1 - 10 N_2 = 10 N_1 N_2 - 10 (N_1 + N_2)
	\]
	арифметических операций. То есть данный алгоритм для перехода на следующий слой занимает порядка $O(N_1 N_2)$ операций. 
	
	\bigskip
	
	
	
	
	

	
	
	\quastion{Почему при увеличении числа измерений резко возрастает количество операций для решения неявных схем (по сравнению с одномерной схемой)?}
	
	Попытка обобщить на случай многих измерений одномерные неявные схемы приводит к необходимости на каждом слое решать систему линейных уравнений большой размерности $N_1 \cdot \ldots \cdot N_p$, где $p$ -- размерность задачи, $N_i$ -- количество шагов в $i$-ом направлении. Будем считать, что шаги во всех направлениях одинаковы, тогда количество уравнений будет равно $N^p$. Главная проблема здесь заключается в том, что матрица этой системы перестает быть трехдиагональной, хотя и сохраняет структуру <<полосы>> с шириной $2N^{p-1}$. В общем случае систему с такой матрицей придется решать методом Гаусса, который потребует порядка $O(N^{3p})$ операций. Даже без учета количества временных слоев это приводит к большим затратам на вычислительные ресурсы, причем в случае $p \geq 2$ расчет по неявной схеме выйдет даже дороже расчета по явной схеме, большое количество операций которой связано с ограничением на шаг по времени: $ \tau \leq \dfrac{1}{2} \left( \dfrac{1}{h_1^2} + \ldots + \dfrac{1}{h_p^2} \right).$
	
	

	\bigskip
	
	
	
	
	
	
	
	
	
	
	
	\quastion{Можно ли использовать метод переменных направлений в
	областях произвольной формы?}
	
	Метод переменных направлений позволяет проводить расчеты на любых выпуклых областях. Если область такая, что можно разделить ее на прямоугольники со сторонами параллельными осям, то, определив граничные условия (идеального контакта на границе) между участками, можно применить данный метод. Также можно рассмотреть разделение области на треугольники. 

	
	
	
	\bigskip
	
	
	\quastion{Можно ли использовать метод переменных направлений
	для решения пространственных и вообще $n$-мерных задач?}
	
	%Метод переменных направлений можно использовать для решения пространственных задач. Рассмотрим трехмерную задачу, тогда метод можно записать в виде:
	%\[
	%\frac{y^{k+\frac{1}{3}} - y^k}{\frac{\tau}{3}} = \Lambda_1 y^{k+\frac{1}{3}} + %\Lambda_2 y^k + \Lambda_3 y^k + \varphi
%	\]
	%Отсюда можно найти $y^{k+\frac{1}{3}}$.
%	\[
%	\frac{y^{k+\frac{2}{3}} - y^{k+\frac{1}{3}}}{\frac{\tau}{3}} = \Lambda_1 y^{k+ \frac{1}{3}} + \Lambda_2 y^{k+ \frac{2}{3}} + \Lambda_3 y^{k + \frac{1}{3}} + \varphi
%	\]
%	Отсюда можно найти $y^{k+\frac{2}{3}}$.
%	\[
%	\frac{y^{k+1} - y^{k + \frac{2}{3}}}{\frac{\tau}{3}} = \Lambda_1 y^{k + \frac{2}{3}} + \Lambda_2 y^{k + \frac{2}{3}} + \Lambda_3 y^{k+1} + \varphi
%	\]
%	Отсюда можно найти $y^{k+1}$.
	
%	Метод не обобщается на случай размерности $n \geqslant 3$, так как возникает несимметричность и условная устойчивость.
	
		
		Метод переменных направлений, описываемый шагами по отдельности не аппроксимирующими исходную дифференциальную задачу, является рабочей схемой
		
		\[
		\frac{3}{\tau}(y^{k+1/3} - y^k) = \Lambda_1 y^{k+1/3} + \Lambda_2 y^{k} + \Lambda_3 y^{k} + \varphi,
		\]
		
		\[
		\frac{3}{\tau}(y^{k+2/3} - y^{k+1/3}) = \Lambda_1 y^{k+1/3} + \Lambda_2 y^{k+2/3} + \Lambda_3 y^{k+1/3} + \varphi,
		\]
		
		\[
		\frac{3}{\tau}(y^{k+1} - y^{k+2/3}) = \Lambda_1 y^{k+2/3} + \Lambda_2 y^{k+2/3} + \Lambda_3 y^{k+1} + \varphi.
		\]
		
		В этой схеме интегрирование по каждому из направлений $x_1, x_2$ и $x_3$ происходит один раз неявным образом и дважды явным. При этом возрастание ошибки в явной схеме не компенсируется её убыванием в неявной, и схема не является безусловно устойчивой.
		
		К задачам большой размерности применяют, например, локально-одномерный метод. Пусть необходимо аппроксимировать уравнение:
		
		\[
		u_t = \Delta u + f, \quad \text{где} \ \Delta u = \sum_{i=1}^p \frac{\partial^2 u}{\partial x_i^2}.
		\]
		
		Между временными слоями $t_k$ и $t_{k+1}$ введем $p+1$ промежуточных слоев с шагом $\tau / p$. Если обозначить решение на промежуточном слое как $\omega_i$, то локально-одномерную схему записывают следующим образом:
		
		\[
		\frac{1}{\tau} (\omega_i - \omega_{i-1}) = \frac{1}{2} \Lambda_i (\omega_i + \omega_{i-1}) + \varphi_i, \quad i = 1, \ldots, p,
		\]
		
		\[
		\omega_0 = y, \quad \omega_p = \hat{y}, \quad \omega_1, \ldots, \omega_{p-1} = \omega.
		\]
		
		Суммарная погрешность аппроксимации такой схемы $O(\tau^2 + \sum_i h_i^2)$.
	
	
	\bigskip
	
	\quastion{Можно ли использовать метод переменных направлений
	на неравномерных сетках?}
	
	Можно, но порядок сходимости по пространству будет ниже. 
	
	Пусть $h_{-1} = x_i - x_{i-1}$, $h_{+1} = x_{i+1} - x_i$. Тогда разложим по формуле Тейлора в $i$-м узле
	\begin{multline*}
		 \Lambda u = \frac{2}{h_{-1} + h_{+1}} (\frac{u_{i+1} - u_i}{h_{+1}} - \frac{u_{i} - u_{i-1}}{h_{-1}}) = \\
		  \frac{2}{h_{-1} + h_{+1}} (\frac{1}{h_{-1}} (u_i + h_{+1} u_x + \frac{h^2_{+1}}{2} u_{xx} + \frac{h^3_{+1}}{6} u_{xxx} + O(h^4_{+1}) - u_i) - \\
		 - \frac{1}{h_{-1}} (u_i - u_i + h_{-1} u_x + \frac{h^2_{-1}}{2} u_{xx} + \frac{h^3_{-1}}{6} u_{xxx} + O(h^4_{-1})) = \\
		 = u_{xx} + \frac{1}{3}u_{xxx}(h_{+1}-h_{-1}) + O(h^2).
	\end{multline*}
	
	Таким образом, на равномерной сетке $h_{-1} = h_{+1} = h$ порядок аппроксимации $O(h^2)$, а на равномерной сетке $O(h_{-1}+h_{+1})$
	
	\begin{equation*} 
	\end{equation*}
\bigskip
	\section{Ответы на дополнительные контрольные вопросы}
	\quastion{Монотонность. Выведите условия положительности коэффициентов для схемы переменных направлений.}
	
	Схема записывается следующим образом:
	\begin{equation*}
		\frac{2}{\tau} y^{k+1/2} - \Lambda_1 y^{k+1/2} = \frac{2}{\tau} y^k + \Lambda_2 y^k + \varphi,
	\end{equation*}
	
	\begin{equation*}
		\frac{2}{\tau} y^{k+1} - \Lambda_2 y^{k+1} = \frac{2}{\tau} y^{k+1/2} + \Lambda_1 y^{k+1/2} + \varphi.
	\end{equation*}
	
	Вычтем первое выражение из второго и выразим $y^{k+1/2}$:
	\begin{equation*}
		\frac{2}{\tau} y^{k+1} - \frac{2}{\tau} y^{k+1/2} - \Lambda_2 y^{k+1} + \Lambda_1 y^{k+1/2} = \frac{2}{\tau} y^{k+1/2} - \frac{2}{\tau} y^k + \Lambda_1 y^{k+1/2} - \Lambda_2 y^k,
	\end{equation*}
	\begin{equation*}
		\frac{2}{\tau} (y^{k+1} + y^k) - \Lambda_2 (y^{k+1} - y^k) = \frac{4}{\tau} y^{k+1/2},
	\end{equation*}
	\begin{equation*}
		y^{k+1/2} = \frac{1}{2} (y^{k+1} + y^k) - \frac{\tau}{4} \Lambda_2 (y^{k+1} - y^k).
	\end{equation*}
	
	Подставим во второе выражение исходной схемы и сократим слагаемые:
	\begin{equation*}
		\frac{1}{\tau} y^{k+1} - \frac{1}{2} \Lambda y^{k+1} + \frac{\tau}{4} \Lambda_1 \Lambda_2 y^{k+1} = \frac{1}{\tau} y^k + \frac{1}{2} \Lambda y^k + \frac{\tau}{4} \Lambda_1 \Lambda_2 y^k + \varphi,
	\end{equation*}
	где $\Lambda = \Lambda_1 + \Lambda_2$. Заметим, что узлы $y^{k+1}_{i+1,j+1}$, $y^{k+1}_{i+1,j-1}$, $y^{k+1}_{i-1,j+1}$ и $y^{k+1}_{i-1,j-1}$ указаны только лишь в слагаемом вида $\Lambda_1 \Lambda_2 y^{k+1}$. Распишем его:
	
	\begin{equation*}
		\Lambda_1 \Lambda_2 y^{k+1} = \Lambda_1 \left(\frac{y^{k+1}_{i,j+1} - 2y^{k+1}_{i,j} + y^{k+1}_{i,j-1}}{h_2^2}\right) = \frac{1}{h_2^2} \left(\Lambda_1 y^{k+1}_{i,j+1} - 2\Lambda_1 y^{k+1}_{i,j} + \Lambda_1 y^{k+1}_{i,j-1}\right) =
	\end{equation*}
	
	\begin{equation*}
		= \frac{1}{h_2^2} \frac{1}{h_1^2} \left( y^{k+1}_{i+1,j+1} - 2y^{k+1}_{i,j+1} + y^{k+1}_{i-1,j+1} - 2(y^{k+1}_{i+1,j} - 2y^{k+1}_{i,j} + y^{k+1}_{i-1,j}) + y^{k+1}_{i+1,j-1} - 2y^{k+1}_{i,j-1} + y^{k+1}_{i-1,j-1} \right).
	\end{equation*}
	
	Видно, что коэффициенты при указанных узлах положительны. Таким образом, после переноса их направо они станут отрицательными. Также можно показать, что коэффициенты при $y^{k+1}_{i,j+1}$, $y^{k+1}_{i,j-1}$, $y^{k+1}_{i-1,j}$ и $y^{k+1}_{i-1,j}$ будут положительными, т.е. условие положительности коэффициентов в любом случае не выполняется.
	
	
	\bigskip
	\quastion{Критерий останова. Какой критерий останова Вы использовали в счёте на установление?}
	
	-
	
	
	
	\quastion{ Трёхмерный случай. Обобщите продольно-поперечную схему на трёхмерный случай. Предоставьте запись схемы.}
	Для многомерного случая обобщением продольно-поперечной схемы является локально-одномерная схема. Рассмотрим уравнение
	\begin{equation*}
		u_t = \sum_{i=1}^n u_{x_i x_i} + f.
	\end{equation*}
	
	Аппроксимируем это уравнение, используя симметричную неявную схему
	\begin{equation*}
		y_t = \sum_{i=1}^n \Lambda_i y^{\frac{1}{2}} + \varphi,
	\end{equation*}
	где $\Lambda_i$ — разностная вторая производная по координате $x_i$.
	
	Наряду с исходной схемой построим локально-одномерную схему. Для этого между слоями $t$ и $t + \tau$ введем $n+1$ промежуточных слоев с шагами $\tau/n$ между ними. Первый слой соответствует моменту времени $t$, последний с номером $n+1$ — моменту времени $t + \tau$. На каждом таком слое с номером $\alpha$ суммарный оператор в правой части заменим оператором $\Lambda_\alpha$. Обозначим решение на промежуточных шагах через $w_\alpha$, $\alpha = 1, 2, \ldots, n$. Тогда $w_\alpha$ является решением следующей разностной задачи:
	\begin{equation*}
		\frac{1}{\tau} (\hat{w}_\alpha - w_\alpha) = \frac{1}{2} \Lambda_\alpha (\hat{w}_\alpha + w_\alpha) + \varphi_\alpha, \quad \alpha = 1, 2, \ldots, n;
	\end{equation*}
	\begin{equation*}
		w_1 = y, \quad w_2 = \hat{w}_1, \ldots, w_n = \hat{w}_{n-1}, \quad \hat{w}_n = y.
	\end{equation*}
	
	Очевидно, что для любого $n$ соответствующее уравнение является одномерным, решаемым методом обычной прогонки. Остальные независимые переменные участвуют в нем только в качестве параметров. Поэтому и схема называется локально-одномерной.
	\bigskip
	
	
	
	\quastion{Порядок сходимости.Покажите, что рассматриваемая схема имеет порядок сходимости, предсказываемый теорией (составьте таблицу погрешностей и порядков сходимости в зависимости от шагов , $h1$ и $h2$).}
	\bigskip
	
	
	\begin{table}[h!]
		\centering
		\adjustbox{scale=0.7}{%
		\begin{tabular}{|c|c|c|c|c|}
			\hline
			$\{h_x, h_y, \tau\}$ & НО на точном решении & Пор. сход. на точном решении & НО по правилу Эйткена & Пор. сход. по пр. Эйткена \\
			\hline
			$\{0.2, 0.2, 0.063662\}$ & $1.7 \cdot 10^{-4}$ & - & - & - \\
			\hline
			$\{0.1, 0.1, 0.031\}$ & $4.1 \cdot 10^{-5}$ & $2.041$ & $5. \cdot 10^{-5}$ & - \\
			\hline
			$\{0.05, 0.05, 0.015\}$ & $1.5\cdot 10^{-5}$ & $1.75$ & $9.49 \cdot 10^{-6}$ & $2.405$ \\
			\hline
		\end{tabular}
	}
		\caption{Порядок метода}
		\label{tab:my_label}
	\end{table}
	
	
	\quastion{ Время выхода на стационар. Как в зависимости от $\tau$ меняется время выхода решения на стационар?}
	\bigskip
	
	
	
	
	
	\clearpage
	\begin{thebibliography}{1}
		\bibitem{1} Галанин М.П., Савенков Е.Б. Методы численного анализа математических моделей. М.: Изд-во МГТУ им. Н.Э. Баумана. 2018. 592 с.
		
	\end{thebibliography}
	
\end{document}